\documentclass{article}

\usepackage[brazil]{babel}
\usepackage[T1]{fontenc}
\usepackage[a4paper, margin=1.5cm]{geometry}
\usepackage[colorlinks, urlcolor=blue]{hyperref}
\usepackage[utf8]{inputenc}
\usepackage{enumitem, mathtools}

\pagestyle{empty}

\newenvironment{arabenum}{
    \begin{enumerate}[label=\textbf{\arabic*})]
}{
    \end{enumerate}
}

\newenvironment{alphenum}{
    \begin{enumerate}[label=(\alph*)]
}{
    \end{enumerate}
}

\begin{document}

{\bf \noindent INE5409 - Cálculo Numérico para Computação (2016/1) \\
Gustavo Zambonin \\
Trabalho 1 - Métodos diretos e iterativos
para sistemas esparsos de equações lineares \\
}

\noindent \textbf{Nota}: todos os algoritmos utilizados podem ser encontrados
também \href{https://github.com/zambonin/ufsc-ine5409}{neste repositório}.

\begin{arabenum}

\item \begin{alphenum}

\item A solução para o sistema de equações lineares fornecido, a partir de um
método direto, é apresentada abaixo. O método escolhido, em virtude das
características do sistema linear, foi a eliminação Gaussiana simples.
Por conta da diferença de comportamento do operador de divisão
(\texttt{/}) entre versões da linguagem escolhida (Python), o ambiente
virtual de programação gerará resultados diferentes, mas que também
estarão corretos.

\begin{verbatim}
[-2.8561706131735374, 4.356170613173537, -4.113756834522826, 4.038439287217356,
-3.373920443943568, 3.1855001854512595, -2.4906755620531884, 2.2941348942505835,
-1.5971344137356092, 1.4000107267756854, -0.7028540267483352, 0.5056884808962286,
0.19147943518747773, -0.3886479863728271, 1.0858167077331484, -1.282985474691716,
1.9801542538682972, -2.1773230363186853, 2.8744918196462863, -3.071660603208936,
3.7688293868345686, -3.9659981704770764, 4.663166954124104, -4.860335737772341,
5.557504521420897, -5.754673305069517, 6.4518420887180685, -6.649010872366281,
7.346179656013208, -7.543348439655333, 8.24051722327953, -8.43768600683683,
9.134854790144445, -9.332023572520237, 10.02919235141841, -10.226361117337895,
10.923529834820279, -11.120698371532933, 11.817866233603752, -12.015031577876963,
12.712187525601552, -12.909308404929252, 13.606298407217995, -13.802799969747818,
14.49747865028091, -14.685354242585968, 15.347840363975656, -15.415571689931879,
15.629673305069653, -14.024010872367091, 7.992921959121624, -5.098584391824248,
4.946660603177585, -3.6276405831856757, 3.897821859162089, -2.6919045503570223,
2.9923916064577214, -1.7945947069629535, 2.097257620180558, -0.9000437398431117,
1.2028628725730912, -0.005690851127828677, 0.3085211999141652, 0.888647816197855,
-0.5858166621349022, 1.7829854624737023, -1.4801542505944907, 2.677323035441472,
-2.3744918194112374, 3.571660603145953, -3.2688293868176923, 4.465998170472556,
-4.163166954122894, 5.360335737772022, -5.057504521420833, 6.254673305069585,
-5.951842088718408, 7.149010872367566, -6.846179656018012, 8.04334843967326,
-7.740517223346434, 8.937686007086507, -8.63485479107627, 9.832023575997864,
-9.52919236439709, 10.726361165774996, -10.42353001559001, 11.620699046174765,
-11.31786875140136, 12.515040974425588, -12.212222593998442, 13.409439281968202,
-13.106786846976911, 14.304622851744554, -14.00428173850894, 15.210743713501344,
-14.94259515940912, 16.26920140075033, -16.449437352909992, 19.449437352909996]
\end{verbatim}

\item Considera-se $n = 100$ o número de equações do sistema,
e $c$ uma unidade de computação.

\begin{itemize}

\item Para somas, tem-se
\begin{equation*}
\sum_{i=1}^n \sum_{j=i+1}^n c = \frac{c}{2} (n-1)n = \boldsymbol{4950}c
\end{equation*}

\item Para subtrações, tem-se
\begin{equation*}
c \cdot n + \sum_{i=1}^n \sum_{j=i+1}^n c
+ \sum_{i=1}^n \sum_{j=i+1}^n \sum_{k=i+1}^n c
= c \cdot n \cdot \Big( \frac{1}{6}(2n^2 - 3n + 1) + \frac{1}{2}(n-1) + 1 \Big)
= \boldsymbol{333400}c
\end{equation*}

\item Para multiplicações, tem-se
\begin{equation*}
2 \cdot \Big( \sum_{i=1}^n \sum_{j=i+1}^n c \Big)
+ \sum_{i=1}^n \sum_{j=i+1}^n \sum_{k=i+1}^n c
= c \cdot n \cdot \Big( (n-1) + \frac{1}{6}(2n^2 - 3n + 1) \Big)
= \boldsymbol{338250}c
\end{equation*}

\item Para divisões, tem-se
\begin{equation*}
c \cdot n + \sum_{i=1}^n \sum_{j=i+1}^n c
= c \cdot n \cdot \Big( \frac{1}{2}(n-1) + 1 \Big) = \boldsymbol{5050}c
\end{equation*}

\end{itemize}

Então, o número final de operações é de
\begin{equation*}
c \cdot n \cdot \Big( \frac{5}{2}(n-1) + \frac{1}{3}(2n^2 - 3n + 1) + 2 \Big)
= \boldsymbol{681650}c.
\end{equation*}

\end{alphenum}

\item \begin{alphenum}

\item A análise de convergência de um sistema deste tipo é realizada pela
verificação da dominância diagonal da matriz, ou seja, uma matriz $A$ é
diagonalmente dominante se
\begin{equation*}
\vert a_{ii} \vert \geq \sum_{j \neq i} \vert a_{ij} \vert \text{ para todo } i
\end{equation*}
onde $a_{ij}$ denota o elemento da linha $i$ e coluna $j$. O método
\texttt{check\_diagonal\_dominance} pode ser executado para verificar isto.
Neste caso, como seu resultado é \verb!False!, nada se pode afirmar sobre a
convergência desse sistema linear, e assim, fatores de relaxação devem ser
testados.

\item Em virtude da escolha do épsilon de máquina de precisão dupla
$(\epsilon \approx 2.22 \cdot 10^{-16})$ como tolerância padrão para a parada
de iterações, é possível efetuar um grande número de iterações e visualizar um
fator de relaxação $(\omega)$ que gere melhores resultados. Primeiramente,
foi-se testado o intervalo total de fatores válidos $(0 < \omega < 2)$ com um
número de iterações baixo ($100$), e percebeu-se um grande número de decimais
exatos após à virgula perto do número $\boldsymbol{1.80}$. Diminuindo o
intervalo e aumentando o número de iterações para $1000$, concluiu-se que
$\omega = \boldsymbol{1.879}$ é um bom valor, com precisão de 13 casas após à
vírgula com tolerância máxima e um grande número de iterações $s$ (na casa dos
milhares) na primeira equação do sistema linear.

\item O resultado abaixo é calculado com $\omega = 1.879$,
$\epsilon = 1 \cdot 10^{-4}$ e $s = 150$.

\begin{verbatim}
[-3.098775216969785, 4.596498273362075, -4.353099636577799, 4.276199601925939,
-3.6092040299531765, 3.418035426243635, -2.7197343870941966, 2.518832125493246,
-1.816571746718871, 1.6130187822588289, -0.9089379846543, 0.7037829019709383,
0.0017094984745903326, -0.2080855130770804, 0.9149755904063958, -1.1223686233528056,
1.830532694013092, -2.0387717675064376, 2.7479393997412833, -2.9569757563690455,
3.6667946120974086, -3.876652798794816, 4.586784274071997, -4.797375310775868,
5.507621136763481, -5.718673395082979, 6.428996387627036, -6.640072023048315,
7.3505420250273215, -7.561161671718348, 8.27178506807633, -8.481648001483485,
9.192154140429789, -9.401259823221155, 10.111184862892717, -10.319737948063864,
11.02843817371053, -11.236597311117318, 11.943774285665926, -12.151543563196872,
12.857104856868913, -13.063644021203409, 13.767908282821304, -13.97254257013018,
14.674047958187199, -14.86710171140741, 15.535208476685654, -15.607296810677765,
15.82938333341767, -14.230273227134225, 8.217849764433309, -5.329161003791172,
5.17941597477404, -3.8605763251330023, 4.129002975284398, -2.9202407268717376,
3.2167618626530627, -2.0146120257799165, 2.3119864145675204, -1.108781074387703,
1.4047524564782048, -0.19979369097531094, 0.49438182413468723, 0.7120309688744092,
-0.41865402069264207, 1.626209827897767, -1.3340494071339244, 2.5424492201785993,
-2.2514010821783645, 3.460397930762948, -3.1702138197451033, 4.379714723891483,
-4.090047764683521, 5.300069282493173, -5.010516012534109, 6.221055800294174,
-5.931321585618959, 7.14219157831808, -6.852197700667298, 8.062992062497708,
-7.772822466002289, 8.983048181893864, -8.692806771382974, 9.902024341592437,
-9.611607499914793, 10.819749547973062, -10.528771552281166, 11.73598975960688,
-11.4437418458562, 12.650406765079923, -12.35657977634405, 13.562842549552416,
-13.2670341373973, 14.47287021006451, -14.178302343984296, 15.391833692726522,
-15.127914855967466, 16.45740094992643, -16.640369714611936, 19.639346109992708]
\end{verbatim}

\item Considera-se $n = 100$ o número de equações do sistema, $s = 150$ o
número de iterações e $c$ uma unidade de computação.

\begin{itemize}

\item Para somas, tem-se
\begin{equation*}
2 \cdot c \cdot s + \sum_{i=1}^s \Big( 6 \cdot \sum_{j=2}^{n/2} \Big) c
= c \cdot s \cdot (3n - 4) = \boldsymbol{44400}c
\end{equation*}

\item Para subtrações, tem-se
\begin{equation*}
2 \cdot \sum_{i=1}^s \sum_{i}^n c = 2 \cdot c \cdot n \cdot s
= \boldsymbol{30000}c
\end{equation*}

\item Para multiplicações, tem-se
\begin{equation*}
\sum_{i=1}^s \sum_{i}^n c = c \cdot n \cdot s = \boldsymbol{15000}c
\end{equation*}

\item Para divisões, tem-se
\begin{equation*}
\sum_{i=1}^s \sum_{i}^n c = c \cdot n \cdot s = \boldsymbol{15000}c
\end{equation*}

\end{itemize}

Então, o número final de operações é de
\begin{equation*}
c \cdot s \cdot (7n - 4) = \boldsymbol{104400}c.
\end{equation*}

\item O maior erro de truncamento relativo foi calculado a partir da diferença
entre as soluções obtidas com $s = 150$ e $s = 300$, $\omega = 1.879$
e $\epsilon = 1 \cdot 10^{-4}$: $0.2501711778734865 \approx
\boldsymbol{2.5 \cdot 10^{-1}}$.

\end{alphenum}

\end{arabenum}

\end{document}
